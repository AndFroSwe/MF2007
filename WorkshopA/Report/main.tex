\documentclass[12pt,a4paper]{article}

\usepackage{amsmath, amssymb}
\usepackage[utf8]{inputenc}
\usepackage[english]{babel}
\usepackage{graphicx}
\usepackage[margin=0.5in]{geometry}
\usepackage{float}

\graphicspath{{fig/}}

\title{MF2007 - Workshop A}

\author{
Adam Lang \\ 861110-3956
\and
Gabriel Andersson Santiago \\ 910706-4538
\and 
Andreas Fr\"oderberg \\ 880730-7577
}

\begin{document}
\maketitle

\section*{Parameter identification}
This section covers the methods behind finding the parameters to enter into the
model to get the biggest similarity to the physical plant.

\subsection*{Level 1}
To identify the parameters of the DC motor, firstly an idealized model is
studied theoretically to understand which physical parameters that affect the
performance of the motor. In the Laplace frequency domain, a model of the motor
without inductance can be expressed as 
\begin{equation}
    \label{eq:motormodel}
    X = \frac {k (1 + \frac{1}{Rd})} {s \frac {J_t} { \frac {k^2} {R} + d} + 1}
    U,
\end{equation}
where $X$ is the rotational speed, $U$ is the input voltage, $k$, $R$ and $d$
are the electric motor constant, internal resistance and the viscous friction.
The variable $J_t$ represents the total inertia of the motor and the load. With
the load inertia after a gear with gear ratio $n$, the total inertia is
\begin{equation}
    \label{eq:inertia}
    J_t = J + \frac{J_{load}}{n^2}.
\end{equation}
Here, the load inertia and the inertia of the motor are separated and the motor
inertia from the datasheet is presumed to be correct, leaving only the load
inertia as an unknown.  Now the relations between the motor parameters and the
time performance can be seen. The time constant is given by the term
\begin{equation}
    \label{eq:timeconstant}
    \frac {J_t} { \frac {k^2} {R} + d}
\end{equation}
and as such contains all the parameters of the motor. According to the final
value theorem, the steady state value from a step input is given by setting
$s=0$ and therefore the steady state gain is given by 
\begin{equation}
    \label{eq:steadystate}
    k (1 + \frac{1}{Rd}).
\end{equation}
This contatiains only one unknown parameter. Therefore, to estimate the viscous
friction parameter $d$, it is suitable to examine and match the steady state
rotational speed of the real system and the model. Since this model contains no
static friction which is probably present in the real system, the test is
conducted at maximum permissable input voltage to minimize the disturbances in
the test data. Since the desired value does not depend on frequency, only a
square wave input is used to simulate steps in both directions. \par
Now that the friction parameter is presumed to be correct and fixed, the time
constant in Equation (\ref{eq:timeconstant}) only has one unknown, the load
inertia. This is estimated from taking the time constant from step inputs,
simulated by a square wave signal. Again, to minimize model disturbances caused
by the static friction present in the real system, the square wave is run at a
full 24 V. 
% INPUT DATA HERE
% ADD DISCUSSION HERE
\subsection*{Level 2}
\label{sub:level_2}
As could be seen in the earlier friction model, it did not incorporate the
effects of static friction which led to inaccuracy in the model when varying the
rotational speed. A model which captures the real behaviour of the system is
used here, called the Karnop friction model. In addition to the linear friction,
the static friction is taken into account by adding a stick-slip zone where the
friction tourque is equal to the applied torque. The model is given by the
equation
\begin{equation}
    \label{eq:karnop}
    M_f = d \dot{\phi} + F_c sgn(\dot{\phi}),
\end{equation}
with the friction torque $M_f$ and the static friction $F_c$. This friction is
the maximum static friction torque that can be applied in the stick slip zone.
There is now one more design parameter that needs to be tuned, $F_c$.
\par
The linear friction is still most prominent when at high velocities and the
static friction is the strongest when at low velocities. The inertia is decided
solely from the rise time and should not differ much from the previous model.
Firstly, the step response to a 5 V square wave is examined to determine the
static friction. When the static friction is set, the response to a square wave
at 24 V is examined and $d$ is adjusted to fit the static friction. A sine of
low amplitude is then run to test the behaviour around the stick slip area, i.e.
at low speeds.

\section{Velocity control}
\label{sec:velocity_control}
The parameters generated in the previous section are used to design the
controller for the motor. For the design, a simplified model without motor
inductance is used in \textsc{Matlab} to place the poles and estimate the
behaviour of the system. Furthermore, a non-linear model with Karnop friction is
used in \textsc{Simulink} to simulate the behaviour of the system with the
controller implemented. Finally, the design is run on the actual motor for
verification of plant model and controller.
\subsection*{Level 1}
\label{sub:velocity_level_1}
Controlling the velocity is done with a PI controller. To best control the
performance, the controller is designed in discrete time. \par
The thumb rule for sample time of a system states that the system should be
sampled at 4-10 times the rise time of the plant. Using the simplified model in
\textsc{Matlab} and the commando \texttt{stepinfo}, the rise time is extracted
and the sampling time is decided. 







\end{document}
