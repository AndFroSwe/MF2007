% vim:tw=72 sw=2 ft=tex
%         File: Part1.tex
% Date Created: 2016 Jan 22
%  Last Change: 2016 Jan 27
%     Compiler: pdflatex
%       Author: Lamn
\documentclass[12pt,a4paper]{article}
\usepackage{amsmath, amssymb}
\usepackage[utf8]{inputenc}
\usepackage[T1]{fontenc}
\usepackage[english]{babel}
\usepackage{graphicx}
\usepackage{mathrsfs}

\graphicspath{{pics/}}

\title{MF2007 Workshop Part1}
\author{Adam Lang, Andreas Fr\"{o}derberg, Gabriel Andersson Santiago}

\newcounter{eq}
\stepcounter{eq}

\newcommand{\eq}[1]{
\begin{equation}
        #1
\end{equation}
    }

\newcommand{\fig}[4]{
  \begin{figure}[!h]
    \begin{center}
      \includegraphics[scale=#1]{#2}
      \label{fig:#3}
      \caption{#4}
    \end{center}
  \end{figure}
}


\begin{document}
\maketitle
\section{Ex. 1}
  \subsection{State Space Model and Transfer Function}
  The model can be described with the following differential equation,
  \eq{
  f-d\dot{x}=m\ddot{x}\Leftrightarrow \ddot{x}=\frac{1}{m}(f-d\dot{x})
  }
  In this system we have one energy storing element, the mass, $m$. The
  transfer function can be derived by first Laplace transforming the
  differential equation,
  \eq{
  \mathscr{L}\{\ddot{x}=\frac{1}{m}(f-d\dot{x})\}\Rightarrow s^2Y=(\frac{1}{m}(U-dsY)
  }
  Then can the transfer function be found as,
  \eq{
  G(s)=\frac{1}{ms^2+ds}
  }
  We can also derive a state space model from the differential equations
  where,
  \eq{
  \begin{bmatrix}
    x_1 \\
    x_2 \\
  \end{bmatrix}
  =
  \begin{bmatrix}
    x \\
    \dot{x} \\
  \end{bmatrix},
  }
  so that, 
  \eq{
  \mathbf{\dot{x}}=
  \begin{bmatrix}
    x_2\\
    \frac{1}{m}(F-dx_2)\\
  \end{bmatrix}.
  }
  This will then give us the state space model as,
  \eq{
  \begin{cases}
  \mathbf{\dot{x}}=
  \begin{bmatrix}
    0 & 1 \\
    0 & -\frac{d}{m} \\
  \end{bmatrix}
  \mathbf{x}+
  \begin{bmatrix}
    0 \\
    \frac{1}{m} \\
  \end{bmatrix}
  \mathbf{u}\\
  \vspace{0.02cm} \\
  \mathbf{y}=
  \begin{bmatrix}
    0 & 1\\
  \end{bmatrix}
  \mathbf{x}
\end{cases}
  }
  \subsection{Matlab and System Characteristics}
    The system is put into MATLAB where the frequency response, the 
    pole zero plot and the step response is plotted, these can be found
    in figure

    \fig{0.5}{plot_step_response.png}{step_response_1}{The step response for
    the system}
    \fig{0.5}{plot_zero_map.png}{pole_zero_2}{The pole zero plot for the
    system}
    \fig{0.5}{plot_bode.png}{frequency_response_1}{The
    frequency response for the system}

  \section{Ex. 3}
    When applying a force to a weightless spring, newtons second law
    ($F=ma$) states that if there is no weight there will be infinite
    acceleration. If we have infinite acceleration with a inertia in
    form of a mass at the other end, there will be full compression of
    the spring infinitely fast. Thus we can model the input force as the
    spring force as input on the mass. We obtain the differential
    equations as
    \eq{
        \begin{cases}
            f=m\dot{v}_2+d\dot{v}_2\\
            f=k(v_2-v_1)
        \end{cases}
    }
    After set to equal and Laplace transformation can the first
    transfer function be obtained as
    \eq{
    G(s)_1=\frac{1}{\frac{m}{s}s^2+\frac{d}{k}s+1}
    }
    and
    \eq{
    G(s)_2=\frac{1}{sm+d}
    }
    which will give the last transfer function as
    \eq{
    G(s)_3=\frac{G(s)_2}{G(s)_1}=\frac{\frac{m}{s}s^2+\frac{d}{k}s+1}{sm+d}
    }
    An improper transfer function represents a pure integration. Till
    will imply that the output is a derivation of the input. If this is
    true, then this will mean that the systems input knows
    future characteristics of the system as implied by the defintion of the 
    derivative $f'(x)=\lim_{h \to 0} \frac{f(x+h)-f(x)}{h}$. Where
    $f(x+h)$ is the future value.\\
    It will be hard, if not impossible, to model a real systems this
    way and one should proceed with caution. 


\end{document}
