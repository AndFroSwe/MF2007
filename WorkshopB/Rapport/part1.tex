%\documentclass[12pt,a4paper]{article}
%
%\usepackage{amsmath, amssymb}
%\usepackage[utf8]{inputenc}
%\usepackage[english]{babel}
%\usepackage{graphicx}
%\usepackage[margin=0.5in]{geometry}
%\usepackage{float}
%
%\graphicspath{{fig/}}
%
%
%\begin{document}
\subsection*{Level 1}
The model following part could be derived by inverting each block of the DC
motor model step by step. This was then verified by using outputs from the DC
motor model as input to the inversed model. This should result in that
the input to the DC motor model is the same as the output for the
inverse model. Since this is the case, this part can be seen as
verified. \\

In order to design the trajectory planner values for $a_{max}$ and
$v_{max}$ needed to be obtained. $v_{max}$ was read from a velocity plot
when the motor model was fed with 24 V. $a_{max}$ was calculated to a
value around 650 \si{\meter\per\square\second} but that value saturated the voltage from the model
follower. $a_{max}$ was therefore tweaked into a value which never
saturates the voltage. These derived values can be seen in
Table~\ref{tbl:table1}.
\begin{table}[H]
    \centering
    \caption{Max values for acceleration and velocity}
	\begin{tabular}{| c | c |}
        \hline
	    $a_{max}$ & 255 \\ \hline
		$v_{max}$ & 270 \\ 
		\hline
	\end{tabular}
        \label{tbl:table1}
\end{table}

The signal from the trajectory planner with Rs = 10 and Rs=100 can be seen in Figure \ref{fig:task2_pos}.

\begin{figure}[H]
	\begin{center}
	
		\includegraphics[width=0.90\linewidth]{task1_traj_rs10.png}
		\caption{The trajectory planner with different Rs values}
		\label{fig:task2_pos}
	\end{center}
\end{figure}


%\end{document}
